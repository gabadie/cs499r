\subsection{Path tracer on a GPU}
Why would we like to implement a Path Tracer on a GPU rather than on a CPU?
The path tracing is shoting a lot of rays with a given random seed
for every pixel. Therefore the convenient thing is you can actually
compute two neighbouring pixels' color completly in parallel for instance.
Nowdays GPUs can run up to thousands threads in parrallel. And even GPU has
its own dedicated thread sheduling system that cam pause threads that are
waiting for a memory fetch for instance, giving the ALUs' compute power to
other thread waiting for available room to continue their execution. But also
this has some slight problems... Video memory is often around 2Gb. Not much if
we want to render a scene as detail as we would see in a movie today. But with
the GPU computing power increasing, considering a real-time path tracing with
simpler scene could very much make sense. This paper is then based in this
very last purpose.

\subsection{A paper dedicated for a GPU implementation}
Why a dedicated paper for Path Tracer implementations for GPU? This question is
can be awnsered with the following question: Why does the GPU has a much bigger
computing power than a CPU? GPUs and CPUs are both made same way:
integrated circuits scratched on wafers and then tiled out
into dies. But there differences come in the design of the transistors. The CPU
will be designed as a general purpose unit, that shall be the master of the
entire computer, giving orders and synchronising everything like a conductor
in an orchestre. Therefore, it needs to be able to do a lot of
different things, from managing the hardware interuptions to the wide range of
instruction set. On the otherhand, the GPU's cores are crowded on the die, thanks
to the fact that
they occupy a smaller size on the die. And since they occupy less space, they
can't do everything a CPU can. For instance, the instruction set of a GPU core
is basically restraint to memory
operations, arithmetic operations and few for logic/branching operations. Nothing else.
No hardware interuptions, no comunication with other component...

\subsection{Difference of a GPU implementation versus a CPU one}
What makes a Path Tracer implementation for GPU so different than for CPU? At
the begining, this project was only intended for studying memory corency on the
GPU, and memory access efficiency.
But it turned out that a bigger problem arises when it comes to a GPU
implementation: this what we will call in this paper the warp efficiency.
