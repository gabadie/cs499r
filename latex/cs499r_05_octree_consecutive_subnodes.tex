To quickly summarize, all octrees nodes (scene's octree and meshes' octree) are
stored as an array in a same GPU buffer. With complex meshes or scenes, we would
then have a lot of node, increasing this buffer size. But when tracing rays
throuht the scene, all thread would be access this buffer quite a couple of time
when browsing each octrees, cloberring the cache. This optimization only changes
the nodes order in that buffer, trying to reduce memory cache's page faults, by
storing all node's sub-nodes consecutively.

\begin{figure}[H]
    \tiny
    \centering
    \begin{tabular}{ | l | c | c | }

        \hline
        Octree consecutive & Stanford dragon & Stanford bunny \\
        subnode optimization & ~ & ~ \\
        \hline
        Disabled & 535.6ns & 116.6ns \\
        Enabled & 527.9ns & 115.1ns \\
        \hline

    \end{tabular}
    \caption{
        Octree consecutive subnode optimization's average rendering time
        per ray.
    }
    \label{table:octree_consecutive_subnode}
\end{figure}
