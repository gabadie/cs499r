Now, we have seen in sub-sections \ref{subsec:cpt} and \ref{subsec:cipt} that
both algorithms are slicing rendering tiles into smaller, cohenercy tiles. And
the warp efficiency is increasing as the parrallel raies are getting closer, since
they have an increasing probability to hit the same triangles. The idea of this
optimization is to get parrallel raies closer, by replacing multi-sampling per
super-sampling. The super-sampling technique is to render into an higher resolution
rendering tile, and then downscales it where it belongs on the final render. This
way, there is more coherency tiles with still the same number of pixels over the
entire rendering viewport. And in our implementation, we actually witness an
improved average rendering time per raies, downscale operation cost included.
But with a given fix number of raies shot over the entire viewport, the image
quality is actually decreasing with CPT and CIPT more than the performance
improvement. This is due to the simple fact less sample passes would then be required
per pixel before being downscaled, and therefore less distinct random seed values per
downscaled pixels.
